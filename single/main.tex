\documentclass[11pt,fleqn]{article}

% ==== 中文環境 ====
\usepackage[nofonts]{ctex}
\usepackage{CJK}
\setCJKmainfont{Microsoft JhengHei}

%% === AMS 標準套件 ===
\usepackage{amsmath,amsfonts,amssymb,amsthm} % 數學符號

%% === Symbol ====
\usepackage{mathtools}
\usepackage{upgreek}
\usepackage{relsize} % 改大小
\usepackage{cancel}	% 刪除線

%% === border ===
\usepackage[margin=1in]{geometry}

%% === TikZ 套件 ===
\usepackage{tikz,tkz-graph,tkz-berge} 

%% 
\setlength{\parindent}{0pt}
\setlength{\parskip}{5pt plus 1pt}
\setlength{\headheight}{13.6pt}
\newtheorem{theorem}{定理}


% start content
\begin{document}\raggedright

\paragraph{} Insert: 6, 3, 2, 1, 5, 4

\begin{align*}
	\left\{\begin{matrix*}[l]
		X_n &:= \text{\# probes per succ search} \\
		Y_n &:= \text{\# probes per unsucc search}
	\end{matrix*}\right.
\end{align*}
\begin{align*}
	\left\{\begin{matrix*}[l]
		E(X) &= \dfrac{1}{6} \left(1+2+3+3+4+4\right) = \dfrac{17}{6} = 1 + \dfrac{\upvartheta(T)}{6} \\
		E(Y) &= \dfrac{1}{7} \left(4+4+3+4+4+3+1\right) = \dfrac{23}{7} = \dfrac{\varepsilon(T)}{7}
	\end{matrix*}\right.
\end{align*}
\begin{align*}
	\left\{\begin{matrix*}[l]
		\upvartheta(T) &:= \mathlarger{\sum}\limits_{x : \text{int node of } T} \text{path length}(x) = 11\\
		\varepsilon(T) &:= \mathlarger{\sum}\limits_{y : \text{ext node of } T} \text{path length}(y) = 23
	\end{matrix*}\right.
\end{align*}
\tikzset{
  treeInode/.style = {align=center, inner sep=0pt, text centered,
    font=\sffamily},
  treeEnode/.style = {align=center, inner sep=0pt, text centered,
    font=\sffamily},
  arn_n/.style = {treeInode, circle, black, draw=black,  
	  text width=1.5em},% arbre rouge noir, noeud noir
  arn_x/.style = {treeEnode, rectangle, draw=black,
    minimum width=1.5em, minimum height=1.5em}% arbre rouge noir, nil
}
\usetikzlibrary{patterns}
\begin{tikzpicture}[-,>=stealth',level/.style={sibling distance = 5cm/#1,
  level distance = 1cm}] 
\node[arn_n]{6}
	child{
		node[arn_n]{3}
		child{
			node[arn_n]{2}
			child{
				node[arn_n]{1}
				child{
					node[arn_x]{0.5}
				}
				child{
					node[arn_x]{1.5}
				}
			}
			child{
				node[arn_x]{2.5}
			}
		}
		child{
			node[arn_n]{5}
			child{
				node[arn_n]{4}
				child{
					node[arn_x]{3.5}
				}
				child{
					node[arn_x]{4.5}
				}
			}
			child{
				node[arn_x]{5.5}
			}
		}
	}
	child{
		node[arn_x]{6.5}
	}
; 
\end{tikzpicture}\begin{tikzpicture}[scale=5]
\draw[->, thick] (-0.1,0) -- (1.1,0);
\foreach \x/\xtext in {0/1,0.2/2,0.4/3,0.6/4,0.8/5,1.0/6}
    \draw[thick] (\x,0.5pt) -- (\x,-0.5pt) node[below] {\xtext};

\draw (0.5,3pt) node[above] {3.5};
\draw (0.5,0.5pt) node[above] {$\downarrow$};
\draw[pattern=north west lines, pattern color=black] (0.4,-.02) rectangle (0.6,0.02);
\end{tikzpicture}

\begin{align*}
	\text{Sample Space} = \left\{\begin{matrix*}[l]
		\Omega_{X_n} &= \left\{(x_1, x_2, \cdots, x_n; k) 
			\left|\begin{matrix}
				(x_1, x_2, \cdots, x_n) \in S_n \\ 
				1 \le k \le n
			\end{matrix}\right. 
		\right\} &, |\Omega_{X_n}| = n \cdot n! \\
		\Omega_{Y_n} &= \left\{(x_1, x_2, \cdots, x_n; y)
	   		\left|\begin{matrix}	
				(x_1, x_2, \cdots, x_n) \in S_n \\
				y = 0.5, \; 1.5, \; \cdots, \; n.5
			\end{matrix}\right.
		\right\} &, |\Omega_{Y_n}| = (n+1)!
	\end{matrix*}\right. 
\end{align*}

\begin{table}[!hb]
	\centering
	\caption{Analysis of Binary Tree Search}
	\label{bst}
	\begin{tabular}{|c | c | c | c | c|}
		\hline
		n & 1 & 2 & 3 &
		\\ \hline
		$E(X_m) = 2 \frac{n+1}{n} H_n - 3$ & 1 & $\frac{3}{2}$ & $\frac{17}{9}$ & $\cdots$ 
		\\ \hline
		$E(Y_m) = 2 H_{n+1} - 2$ & 1 & $\frac{5}{3}$ & $\frac{13}{6}$ & $\cdots$ 
		\\ \hline
	\end{tabular}
\end{table}

\begin{description}
	\item[\framebox{$n=1$}]
		\tikzset{
  treeInode/.style = {align=center, inner sep=0pt, text centered,
    font=\sffamily},
  treeEnode/.style = {align=center, inner sep=0pt, text centered,
    font=\sffamily},
  arn_n/.style = {treeInode, circle, black, draw=black, fill=black, 
	  text width=0.5em},% arbre rouge noir, noeud noir
  arn_x/.style = {treeEnode, rectangle, draw=black,
    minimum width=0.5em, minimum height=0.5em}% arbre rouge noir, nil
}
\usetikzlibrary{patterns}
\begin{tikzpicture}[-,>=stealth',level/.style={sibling distance = 1cm/#1,
  level distance = 0.5cm}] 
\node[arn_n]{}
	child{
		node[arn_x]{}
	}
	child{
		node[arn_x]{}
	}
; 
\end{tikzpicture}

		\begin{align*}
			\left\{\begin{matrix*}[l]
				E(X_1) &= \frac{1}{1} \left[1\right] = 1 \\
				E(Y_1) &= \frac{1}{2} \left[1+1\right] = 1 \\
			\end{matrix*}\right.
		\end{align*}
	\item[\framebox{$n=2$}]
		\tikzset{
  treeInode/.style = {align=center, inner sep=0pt, text centered,
    font=\sffamily},
  treeEnode/.style = {align=center, inner sep=0pt, text centered,
    font=\sffamily},
  arn_n/.style = {treeInode, circle, black, draw=black, fill=black, 
	  text width=0.5em},% arbre rouge noir, noeud noir
  arn_x/.style = {treeEnode, rectangle, draw=black,
    minimum width=0.5em, minimum height=0.5em}% arbre rouge noir, nil
}
\usetikzlibrary{patterns}

\begin{tikzpicture}[-,>=stealth',level/.style={sibling distance = 1cm/#1,
  level distance = 0.5cm}] 
\node[arn_n]{}
	child{
		node[arn_x]{}
	}
	child{
		node[arn_n]{}
    child{
      node[arn_x]{}
    }
    child{
      node[arn_x]{}
    }
	}
; 
\node[below=1.5cm, align=flush center,text width=2cm]{$(1, 2)$};
\end{tikzpicture}
\qquad
\begin{tikzpicture}[-,>=stealth',level/.style={sibling distance = 1cm/#1,
  level distance = 0.5cm}] 
\node[arn_n]{}
  child{
    node[arn_n]{}
    child{
      node[arn_x]{}
    }
    child{
      node[arn_x]{}
    }
  }
  child{
    node[arn_x]{}
  }
; 
\node[below=1.5cm, align=flush center,text width=2cm]{$(2, 1)$};
\end{tikzpicture}

		\begin{align*}
			\left\{\begin{matrix*}[l]
				E(X_2) &= \frac{1}{2 \cdot 2!} \left[(1+2) + (1+2)\right] 
						= \frac{3}{2} \\
				E(Y_2) &= \frac{1}{3 \cdot 3!} \left[(1+2+2) + (2+2+1)\right] 
						= \frac{5}{3}
			\end{matrix*}\right.
		\end{align*}
	\item[\framebox{$n=3$}]
		\tikzset{
  treeInode/.style = {align=center, inner sep=0pt, text centered,
    font=\sffamily},
  treeEnode/.style = {align=center, inner sep=0pt, text centered,
    font=\sffamily},
  arn_n/.style = {treeInode, circle, black, draw=black, fill=black, 
	  text width=0.5em},% arbre rouge noir, noeud noir
  arn_x/.style = {treeEnode, rectangle, draw=black,
    minimum width=0.5em, minimum height=0.5em}% arbre rouge noir, nil
}
\usetikzlibrary{patterns}

\begin{tikzpicture}[-,>=stealth',level/.style={sibling distance = 1cm/#1,
  level distance = 0.5cm}] 
\node[arn_n]{}
  child{
    node[arn_x]{}
  }
  child{
    node[arn_n]{}
    child{
      node[arn_x]{}
    }
    child{
      node[arn_n]{}
      child{
        node[arn_x]{}
      }
      child{
        node[arn_x]{}
      }
    }
  }
; 
\node[below=2.0cm, align=flush center,text width=2cm]{$(1, 2, 3)$};
\end{tikzpicture}
\begin{tikzpicture}[-,>=stealth',level/.style={sibling distance = 1cm/#1,
  level distance = 0.5cm}] 
\node[arn_n]{}
  child{
    node[arn_x]{}
  }
  child{
    node[arn_n]{}
    child{
      node[arn_n]{}
      child{
        node[arn_x]{}
      }
      child{
        node[arn_x]{}
      }
    }
    child{
      node[arn_x]{}
    }
  }
; 
\node[below=2.0cm, align=flush center,text width=2cm]{$(1, 3, 2)$};
\end{tikzpicture}
\begin{tikzpicture}[-,>=stealth',level/.style={sibling distance = 1cm/#1,
  level distance = 0.5cm}] 
\node[arn_n]{}
  child{
    node[arn_n]{}
    child{
      node[arn_x]{}
    }
    child{
      node[arn_x]{}
    }
  }
  child{
    node[arn_n]{}
    child{
      node[arn_x]{}
    }
    child{
      node[arn_x]{}
    }
  }
; 
\node[below=2.0cm, align=flush center,text width=2cm]{$(2, 1, 3)$};
\end{tikzpicture}
\begin{tikzpicture}[-,>=stealth',level/.style={sibling distance = 1cm/#1,
  level distance = 0.5cm}] 
\node[arn_n]{}
  child{
    node[arn_n]{}
    child{
      node[arn_x]{}
    }
    child{
      node[arn_x]{}
    }
  }
  child{
    node[arn_n]{}
    child{
      node[arn_x]{}
    }
    child{
      node[arn_x]{}
    }
  }
; 
\node[below=2.0cm, align=flush center,text width=2cm]{$(2, 3, 1)$};
\end{tikzpicture}
\begin{tikzpicture}[-,>=stealth',level/.style={sibling distance = 1cm/#1,
  level distance = 0.5cm}] 
\node[arn_n]{}
  child{
    node[arn_n]{}
    child{
      node[arn_x]{}
    }
    child{
      node[arn_n]{}
      child{
        node[arn_x]{}
      }
      child{
        node[arn_x]{}
      }
    }
  }
  child{
    node[arn_x]{}
  }
; 
\node[below=2.0cm, align=flush center,text width=2cm]{$(3, 1, 2)$};
\end{tikzpicture}
\begin{tikzpicture}[-,>=stealth',level/.style={sibling distance = 1cm/#1,
  level distance = 0.5cm}] 
\node[arn_n]{}
  child{
    node[arn_n]{}
    child{
      node[arn_n]{}
      child{
        node[arn_x]{}
      }
      child{
        node[arn_x]{}
      }
    }
    child{
      node[arn_x]{}
    }
  }
  child{
    node[arn_x]{}
  }
; 
\node[below=2.0cm, align=flush center,text width=2cm]{$(3, 2, 1)$};
\end{tikzpicture}

		\begin{align*}
			\left\{\begin{matrix*}[l]
				E(X_3) &= \frac{1}{3 \cdot 3!} \left[(1+2+3) + (1+2+3) +
							(1+2+2) + (\cdot) + (\cdot) + (\cdot) \right]
						= \frac{17}{9} \\
				E(Y_3) &= \frac{1}{4 \cdot 3!} \left[(1+2+3+3) + (1+2+3+3) +
							(2+2+2+2) + (\cdot) + (\cdot) + (\cdot) \right] 
						= \frac{13}{6}
			\end{matrix*}\right.
		\end{align*}
\end{description}

\begin{theorem}
$n$-node binary tree $T$ has $2N$ edges, $N+1$ external nodes, and $\varepsilon(T)=\upvartheta(T)+2N$. 
\end{theorem}

\begin{theorem}
\begin{align*}
	\left\{\begin{matrix*}[l]
		E(X_n) &= 2 \frac{n+1}{n} H_n - 3 &\approx 2 \ln n \\
		E(Y_n) &= 2 H_{n+1} - 2 &\approx 2 \ln n
	\end{matrix*}\right.
\end{align*}
\begin{align*}
	\left\{\begin{matrix*}[l]
		E(X_n) 
			&= \sum\limits_{w \in \Omega_{X_n} } P(w) X_n(w) 
				= \sum\limits_{
						\substack{
						(x_1, \cdots, x_n) \in S_n \\ 
						1 \le k \le n
						}} 
					\frac{1}{n \cdot n!} X_n (x_1, \cdots, x_n ; k) \\
			&= \frac{1}{n \cdot n!} 
				\sum\limits_{(x_1, \cdots, x_n) \in S_n} \sum\limits_{1 \le k \le n} 
					X_n (x_1, \cdots, x_n ; k) \\
			&= \frac{1}{n \cdot n!} \sum\limits_{T} \left(n + \upvartheta(T) \right) 
				= \frac{1}{n} 
					\left[ n + \frac{1}{n!} \sum\limits_{T} \upvartheta(T)\right]\\
		E(Y_n) 
			&= \frac{1}{(n+1)!} \sum\limits_{(x_1, \cdots, x_n) \in S_n}
					\sum\limits_{y = 0.5, \cdots, n.5} Y_n(x_1 , \cdots, x_n ; y) \\
			&= \frac{1}{(n+1)!} \sum\limits_{T} \varepsilon(T) 
				= \frac{1}{(n+1) \cdot n!} \sum\limits_{T} \left(2n + \upvartheta(T)\right) \\
			&= \frac{1}{n+1} \left[2n + \frac{1}{n!} \sum\limits_{T} \upvartheta(T)\right]
	\end{matrix*}\right.
\end{align*}
\end{theorem}

\begin{align*}
	(n+1) E(Y_n) &= \sum\limits_{1 \le k \le n} E(Y_{k-1}) + 2n 
		&& \text{(1)} \\
	n E(Y_{n-1}) &= \sum\limits_{1 \le k \le n-1} E(Y_{k-1}) + 2(n-1)
		&& \text{(2)} \\
	\text{(1)} - \text{(2)} &=\\
		& \qquad
		\begin{matrix*}[l]
		 	(n+1) &E(Y_n) &=& (n+1) E(Y_{n-1}) + 2 \\
			&E(Y_n) &=& \cancel{E(Y_{n-1})} + \frac{2}{n+1} \\
			& & \vdots &\\
			&\cancel{E(Y_2)} &=& E(Y_1) + \frac{2}{3}
		\end{matrix*} \\
	&\Rightarrow
		\left\{\begin{matrix*}[l]
			E(Y_n) &= 2 H_{n+1} - 2 \\
			E(X_n) &= \frac{n+1}{n} E(Y_{n}) - 1 = 2 \frac{n+1}{n} H_n - 3
		\end{matrix*}\right.
\end{align*}

\begin{align*}
E(X_n) 
	&= \frac{1}{n \cdot n!} \sum\limits_{(x_1, \cdots, x_n)} 
		\sum_{1 \le k \le n} X_n (x_1, \cdots, x_n ; k) \\
	&= \frac{1}{n \cdot n!} \sum\limits_{(x_1, \cdots, x_n) \in S_n}
		\sum\limits_{1 \le k \le n} 1 + Y_n (x_1, \cdots, x_{k-1} ; k) \\
	&= 1 + \frac{1}{n \cdot n!} \sum\limits_{1 \le k \le n}
		\sum\limits_{(x_1, \cdots, x_{k}) \in S_k} 
			n \cdot (k+1) Y_{k-1} (x_1, \cdots, x_{k-1}; x_k) \\
	&= 1 + \frac{1}{n} \sum\limits_{1 \le k \le n}
		\sum\limits_{(x_1, \cdots, x_{k}) \in S_k} 
			Y_{k-1} (x_1, \cdots, x_{k-1}; x_k) \\
	&= 1 + \frac{1}{n} \sum\limits_{1 \le k \le n} E(Y_{k-1})
\end{align*}

\end{document}