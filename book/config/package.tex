%% === CJK 套件 ===
\usepackage[nofonts]{ctex}
\usepackage{CJK,CJKnumb}                 % 中文套件
%\setCJKmainfont{Microsoft JhengHei}
%\setCJKmainfont{Heiti TC}
\setCJKmainfont[BoldFont={SimHei},ItalicFont={[simkai.ttf]}]{SimSun}
\setCJKsansfont{SimHei}
\setCJKmonofont{[simfang.ttf]}
 
\setCJKfamilyfont{zhsong}{SimSun}
\setCJKfamilyfont{zhhei}{SimHei}
\setCJKfamilyfont{zhkai}{[simkai.ttf]}
\setCJKfamilyfont{zhfs}{[simfang.ttf]}
\setCJKmainfont{Heiti TC}

%% === AMS 標準套件 ===
\usepackage{amsmath,amsfonts,amssymb,amsthm} % 數學符號

%% === Symbol ====
\usepackage{mathtools}
\usepackage{upgreek}

%% === border ===
\usepackage[margin=1in]{geometry}
%% === header ===
\usepackage{fancyhdr}
\pagestyle{fancyplain}

%% === ===
\usepackage{algorithm}
\usepackage{listings}                        % 程式碼

%% === TikZ 套件 ===
\usepackage{tikz,tkz-graph,tkz-berge}        % 繪圖
\usepackage{multicol}

%% == ==
\usepackage[unicode]{hyperref}
\usepackage{xcolor}
\hypersetup{
    colorlinks,
    linkcolor={blue!100!black},
    citecolor={blue!75!black},
    urlcolor={blue!50!black}
}

%% == 調整設定 ==
\usepackage{enumitem}                           % 修改 enumerate, item
\usepackage{bbding}
\usepackage{titletoc,titlesec,imakeidx}

%% == ==
\usepackage{newfloat}
\usepackage{caption,subcaption}
\usepackage{xkeyval,xargs}
\usepackage{ulem}
\usepackage{import}

%% === 設定 C++ 格式 ===
\lstset{%
  language=C++,             % 設定語言
  %% === 空白, tab 相關 ===
  tabsize=2,                % 設定 tab = 多少空白
  %showspaces=true,          % 設定是否標示空白
  %showtabs=true,            % 設定是否標示 tab
  %tab=\rightarrowfill,      % 設定 tab 樣式
  %% === 行數相關 ===
  numbers=left,             % 行數標示位置
  stepnumber=1,             % 每隔幾行標示行數
  numberstyle=\tiny,
  %breaklines=true,          % 設定斷行
  %% === 顏色設定 ===
  basicstyle=\ttfamily,
  keywordstyle=\color{blue}\ttfamily,
  stringstyle=\color{red!50!brown}\ttfamily,
  commentstyle=\color{green!50!black}\ttfamily,
  %identifierstyle=\color{black}\ttfamily,
  emphstyle=\color{purple}\ttfamily,
  extendedchars=false,
  texcl=true,
  moredelim=[l][\color{magenta}]{\#},
  captionpos=b,
  %% === 其他 ===
  %frame=single
}

% ===============================================
%
%  設定頁面格式
%
% ===============================================
%% === 設定頁面格式 ===
\setlength{\parindent}{0pt}
\setlength{\parskip}{5pt plus 1pt}
\setlength{\headheight}{13.6pt}
%\hoffset         = 10pt                      % 水平位移,預設為 0pt
%\voffset         = -15pt                     % 垂直位移,預設為 0pt
%\oddsidemargin   = 0pt                       % 預設為 31pt
%\topmargin       = 20pt                      % 預設為 20pt
%\headheight      = 12pt                      % header 的高度,預設為 12pt
%\headsep         = 25pt                      % header 和 body 的距離,預設為 25pt
%\textheight      = 620pt                     % body 內文部分的高度,預設為 592pt
%\textwidth       = 450pt                     % body 內文部分的寬度,預設為 390pt
%\marginparsep    = 10pt                      % margin note 和 body 的距離,預設為 10pt
%\marginparwidth  = 35pt                      % margin note 的寬度,預設為 35pt
%\footskip        = 30pt                      % footer 高度 + footer 和 body 的距離,預設為 30pt

\usetikzlibrary{arrows}

%% ==  ==

\DeclareFloatingEnvironment[fileext=frm,placement={!ht},name=Frame]{code}
\captionsetup[code]{labelfont=bf}

\makeindex

\linespread{1.14}
